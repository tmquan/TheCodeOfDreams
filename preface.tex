%=====================================================================================================
\lettrine[lines=3]{\initfamily\textcolor{darkgreen}{C}}{uốn} sách mà bạn sắp đọc tới đây là một thứ hỗn độn muôn màu nhiều vẻ. 
Vì nó đang dẫn bạn vào câu chuyện của mật mã mộng mị mà. 
Như những tia sét xé toạc bầu trời để tìm đến mặt đất, giấc mơ hay mộng mị cũng đưa chúng ta từ đầu cơn buồn ngủ cho đến lúc thức dậy bằng các con đường khác nhau. 
Có lúc nhanh, lúc chậm, có lúc dồn dập, khi lại êm đềm. 
``Mật mã mộng mị'' cho bạn một cái nhìn khác so với các nghiên cứu về tâm thần học trước đây. 
Tất nhiên nó không chính xác hoàn toàn vì là cảm nhận của cá nhân tớ. 
Nếu bạn đọc tới đây mà thấy chán thì nên gấp cuốn sách lại cho phẳng phiu, để nó được như mới nhé. 
Hoặc nếu bạn đang đọc bằng tập tin thì có thể đóng nó lại bằng cách gì cũng được, tùy vào thiết bị bạn đang dùng là máy tính hay điện thoại. 
Còn nếu bạn vẫn liếc mắt cho tới dòng này thì... chúng ta bắt đầu mơ nào. 

Mà khoan, xin trân trọng nhắc trước là nội dung ngôn từ trong cuốn sách này cũng giống như một giấc mơ. 
Tức là cũng hỗn độn và đa hình, lúc là nhận định của một đứa trẻ thơ, lúc là suy nghĩ của một nhà khoa học tập sự. 
Cũng có lúc là thuở mới yêu với những rung động ban đầu, hoặc là xót xa buồn đau khi một mối tình tan vỡ. 
Đây không hẳn là một cuốn sách tiên tri trước tương lai, cũng không phải là cẩm nang giải điềm đoán mộng. 
Nhưng có thể bạn sẽ thấy được những thứ mà chính bạn đã từng trải nghiệm cho đến lúc này. 
Và nếu bạn đã sẵn sàng, tại sao ta không chuẩn bị mơ và mơ đẹp cho đến khi thức dậy? 


